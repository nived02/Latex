\documentclass[11pt]{article}
\usepackage{amsmath}
\usepackage{amsfonts}
\usepackage{geometry}
\usepackage{fancyhdr}
\usepackage{enumitem}

\geometry{margin=1in}
\pagestyle{fancy}
\fancyhf{}
\fancyhead[L]{Mathematical and Numerical Methods for Biologists}
\fancyhead[R]{BB 523}
\fancyfoot[C]{\thepage}

\title{Numerically computing derivatives,Taylor series,Newton-Raphson method}
\author{}
\begin{document}

\maketitle
\thispagestyle{fancy}

\begin{enumerate}
    % Taylor's Series Questions
    \item \textbf{Taylor's Series}
    \begin{enumerate}
        \item Using Taylor's series, expand \( e^x \) till the 3rd order.
        
        \item Using Taylor's series, expand \( e^{-x} \) till the 3rd order.
        
        \item Using Taylor's series, expand \( -\cos(x) \) till the 3rd order.
        
        \item Evaluate the Taylor Series for \( f(x) = x^3 - 10x^2 + 6 \) at \( x = 3 \).
        
        \item Consider a simple population growth model where the population \( P(t) \) grows according to the exponential function \( P(t) = P_0 e^{rt} \), where \( P_0 \) is the initial population size, \( r \) is the growth rate, and \( t \) is time. Given \( P_0 = 100 \), \( r = 0.1 \), and \( t = 2 \), find the approximate population using the Taylor series.
    \end{enumerate}

    % Newton-Raphson Method Questions
    \item \textbf{Newton-Raphson Method}
    \begin{enumerate}
        \item The equation \( x^3 - x^2 + 4x - 4 = 0 \) is to be solved using the Newton-Raphson method. If \( x = 2 \) is taken as the initial approximation, what will be the next approximation?

        \item Solve the equation \( e^x - 1 = 0 \) using Newton's method with an initial guess \( x_0 = -1 \). Estimate \( x_1 \) of the solution.

        \item The Newton-Raphson method is used to compute a root of the equation \( x^2 - 13 = 0 \), with an initial value of 3.5. What will be the approximation after one iteration?

        \item Use the Newton-Raphson method to find the root of the function \( f(x) = x^2 - 2x \). Give values till \( x_3 \), starting from an initial value of \( x_0 = 1.5 \).

        \item The Newton-Raphson method is used to find the roots of the equation \( x^3 + 2x^2 + 3x - 1 = 0 \). If the initial guess is \( x_0 = 1 \), what is the value of \( x \) after the second iteration?
    \end{enumerate}

    % Numerically Computing Derivatives Questions
    \item \textbf{Numerically Computing Derivatives}
    \begin{enumerate}
        \item Consider a population \( P(t) \) that grows exponentially according to the equation \( P(t) = P_0 e^{rt} \), where \( P_0 = 100 \) (initial population), \( r = 0.1 \), and \( t \) is time. Find the rate of population growth \( P'(t) \) at \( t = 2 \) using numerical differentiation methods.
        \begin{enumerate}
            \item Use forward difference approximation.
            \item Use backward difference approximation.
            \item Use central difference approximation.
        \end{enumerate}

        \item Use the forward difference approximation to find the derivative of the function \( f(x) = x^2 - 2x \) at \( x_0 = 1.5 \). Give the value of \( f'(x) \).
    \end{enumerate}

\end{enumerate}


\end{document}
